% Options for packages loaded elsewhere
\PassOptionsToPackage{unicode}{hyperref}
\PassOptionsToPackage{hyphens}{url}
%
\documentclass[
]{article}
\usepackage{amsmath,amssymb}
\usepackage{iftex}
\ifPDFTeX
  \usepackage[T1]{fontenc}
  \usepackage[utf8]{inputenc}
  \usepackage{textcomp} % provide euro and other symbols
\else % if luatex or xetex
  \usepackage{unicode-math} % this also loads fontspec
  \defaultfontfeatures{Scale=MatchLowercase}
  \defaultfontfeatures[\rmfamily]{Ligatures=TeX,Scale=1}
\fi
\usepackage{lmodern}
\ifPDFTeX\else
  % xetex/luatex font selection
\fi
% Use upquote if available, for straight quotes in verbatim environments
\IfFileExists{upquote.sty}{\usepackage{upquote}}{}
\IfFileExists{microtype.sty}{% use microtype if available
  \usepackage[]{microtype}
  \UseMicrotypeSet[protrusion]{basicmath} % disable protrusion for tt fonts
}{}
\makeatletter
\@ifundefined{KOMAClassName}{% if non-KOMA class
  \IfFileExists{parskip.sty}{%
    \usepackage{parskip}
  }{% else
    \setlength{\parindent}{0pt}
    \setlength{\parskip}{6pt plus 2pt minus 1pt}}
}{% if KOMA class
  \KOMAoptions{parskip=half}}
\makeatother
\usepackage{xcolor}
\usepackage[margin=1in]{geometry}
\usepackage{color}
\usepackage{fancyvrb}
\newcommand{\VerbBar}{|}
\newcommand{\VERB}{\Verb[commandchars=\\\{\}]}
\DefineVerbatimEnvironment{Highlighting}{Verbatim}{commandchars=\\\{\}}
% Add ',fontsize=\small' for more characters per line
\usepackage{framed}
\definecolor{shadecolor}{RGB}{248,248,248}
\newenvironment{Shaded}{\begin{snugshade}}{\end{snugshade}}
\newcommand{\AlertTok}[1]{\textcolor[rgb]{0.94,0.16,0.16}{#1}}
\newcommand{\AnnotationTok}[1]{\textcolor[rgb]{0.56,0.35,0.01}{\textbf{\textit{#1}}}}
\newcommand{\AttributeTok}[1]{\textcolor[rgb]{0.77,0.63,0.00}{#1}}
\newcommand{\BaseNTok}[1]{\textcolor[rgb]{0.00,0.00,0.81}{#1}}
\newcommand{\BuiltInTok}[1]{#1}
\newcommand{\CharTok}[1]{\textcolor[rgb]{0.31,0.60,0.02}{#1}}
\newcommand{\CommentTok}[1]{\textcolor[rgb]{0.56,0.35,0.01}{\textit{#1}}}
\newcommand{\CommentVarTok}[1]{\textcolor[rgb]{0.56,0.35,0.01}{\textbf{\textit{#1}}}}
\newcommand{\ConstantTok}[1]{\textcolor[rgb]{0.00,0.00,0.00}{#1}}
\newcommand{\ControlFlowTok}[1]{\textcolor[rgb]{0.13,0.29,0.53}{\textbf{#1}}}
\newcommand{\DataTypeTok}[1]{\textcolor[rgb]{0.13,0.29,0.53}{#1}}
\newcommand{\DecValTok}[1]{\textcolor[rgb]{0.00,0.00,0.81}{#1}}
\newcommand{\DocumentationTok}[1]{\textcolor[rgb]{0.56,0.35,0.01}{\textbf{\textit{#1}}}}
\newcommand{\ErrorTok}[1]{\textcolor[rgb]{0.64,0.00,0.00}{\textbf{#1}}}
\newcommand{\ExtensionTok}[1]{#1}
\newcommand{\FloatTok}[1]{\textcolor[rgb]{0.00,0.00,0.81}{#1}}
\newcommand{\FunctionTok}[1]{\textcolor[rgb]{0.00,0.00,0.00}{#1}}
\newcommand{\ImportTok}[1]{#1}
\newcommand{\InformationTok}[1]{\textcolor[rgb]{0.56,0.35,0.01}{\textbf{\textit{#1}}}}
\newcommand{\KeywordTok}[1]{\textcolor[rgb]{0.13,0.29,0.53}{\textbf{#1}}}
\newcommand{\NormalTok}[1]{#1}
\newcommand{\OperatorTok}[1]{\textcolor[rgb]{0.81,0.36,0.00}{\textbf{#1}}}
\newcommand{\OtherTok}[1]{\textcolor[rgb]{0.56,0.35,0.01}{#1}}
\newcommand{\PreprocessorTok}[1]{\textcolor[rgb]{0.56,0.35,0.01}{\textit{#1}}}
\newcommand{\RegionMarkerTok}[1]{#1}
\newcommand{\SpecialCharTok}[1]{\textcolor[rgb]{0.00,0.00,0.00}{#1}}
\newcommand{\SpecialStringTok}[1]{\textcolor[rgb]{0.31,0.60,0.02}{#1}}
\newcommand{\StringTok}[1]{\textcolor[rgb]{0.31,0.60,0.02}{#1}}
\newcommand{\VariableTok}[1]{\textcolor[rgb]{0.00,0.00,0.00}{#1}}
\newcommand{\VerbatimStringTok}[1]{\textcolor[rgb]{0.31,0.60,0.02}{#1}}
\newcommand{\WarningTok}[1]{\textcolor[rgb]{0.56,0.35,0.01}{\textbf{\textit{#1}}}}
\usepackage{longtable,booktabs,array}
\usepackage{calc} % for calculating minipage widths
% Correct order of tables after \paragraph or \subparagraph
\usepackage{etoolbox}
\makeatletter
\patchcmd\longtable{\par}{\if@noskipsec\mbox{}\fi\par}{}{}
\makeatother
% Allow footnotes in longtable head/foot
\IfFileExists{footnotehyper.sty}{\usepackage{footnotehyper}}{\usepackage{footnote}}
\makesavenoteenv{longtable}
\usepackage{graphicx}
\makeatletter
\def\maxwidth{\ifdim\Gin@nat@width>\linewidth\linewidth\else\Gin@nat@width\fi}
\def\maxheight{\ifdim\Gin@nat@height>\textheight\textheight\else\Gin@nat@height\fi}
\makeatother
% Scale images if necessary, so that they will not overflow the page
% margins by default, and it is still possible to overwrite the defaults
% using explicit options in \includegraphics[width, height, ...]{}
\setkeys{Gin}{width=\maxwidth,height=\maxheight,keepaspectratio}
% Set default figure placement to htbp
\makeatletter
\def\fps@figure{htbp}
\makeatother
\setlength{\emergencystretch}{3em} % prevent overfull lines
\providecommand{\tightlist}{%
  \setlength{\itemsep}{0pt}\setlength{\parskip}{0pt}}
\setcounter{secnumdepth}{-\maxdimen} % remove section numbering
\ifLuaTeX
  \usepackage{selnolig}  % disable illegal ligatures
\fi
\IfFileExists{bookmark.sty}{\usepackage{bookmark}}{\usepackage{hyperref}}
\IfFileExists{xurl.sty}{\usepackage{xurl}}{} % add URL line breaks if available
\urlstyle{same}
\hypersetup{
  pdftitle={Halloween Candy},
  pdfauthor={Alex Cagle},
  hidelinks,
  pdfcreator={LaTeX via pandoc}}

\title{Halloween Candy}
\author{Alex Cagle}
\date{2023-05-11}

\begin{document}
\maketitle

\begin{Shaded}
\begin{Highlighting}[]
\NormalTok{candy\_file }\OtherTok{\textless{}{-}} \StringTok{"candy{-}data.csv"}

\NormalTok{candy }\OtherTok{=} \FunctionTok{read.csv}\NormalTok{(candy\_file, }\AttributeTok{row.names=}\DecValTok{1}\NormalTok{)}
\FunctionTok{head}\NormalTok{(candy)}
\end{Highlighting}
\end{Shaded}

\begin{verbatim}
##              chocolate fruity caramel peanutyalmondy nougat crispedricewafer
## 100 Grand            1      0       1              0      0                1
## 3 Musketeers         1      0       0              0      1                0
## One dime             0      0       0              0      0                0
## One quarter          0      0       0              0      0                0
## Air Heads            0      1       0              0      0                0
## Almond Joy           1      0       0              1      0                0
##              hard bar pluribus sugarpercent pricepercent winpercent
## 100 Grand       0   1        0        0.732        0.860   66.97173
## 3 Musketeers    0   1        0        0.604        0.511   67.60294
## One dime        0   0        0        0.011        0.116   32.26109
## One quarter     0   0        0        0.011        0.511   46.11650
## Air Heads       0   0        0        0.906        0.511   52.34146
## Almond Joy      0   1        0        0.465        0.767   50.34755
\end{verbatim}

\begin{quote}
Q1. How many different candy types are in this dataset?
\end{quote}

\begin{Shaded}
\begin{Highlighting}[]
\FunctionTok{nrow}\NormalTok{(candy)}
\end{Highlighting}
\end{Shaded}

\begin{verbatim}
## [1] 85
\end{verbatim}

\begin{quote}
Q2. How many fruity candy types are in the dataset?
\end{quote}

\begin{Shaded}
\begin{Highlighting}[]
\FunctionTok{sum}\NormalTok{(candy}\SpecialCharTok{$}\NormalTok{fruity)}
\end{Highlighting}
\end{Shaded}

\begin{verbatim}
## [1] 38
\end{verbatim}

\begin{quote}
Q: What are these fruity candy?
\end{quote}

\begin{Shaded}
\begin{Highlighting}[]
\FunctionTok{rownames}\NormalTok{(candy[candy}\SpecialCharTok{$}\NormalTok{fruity }\SpecialCharTok{==} \DecValTok{1}\NormalTok{, ])}
\end{Highlighting}
\end{Shaded}

\begin{verbatim}
##  [1] "Air Heads"                   "Caramel Apple Pops"         
##  [3] "Chewey Lemonhead Fruit Mix"  "Chiclets"                   
##  [5] "Dots"                        "Dum Dums"                   
##  [7] "Fruit Chews"                 "Fun Dip"                    
##  [9] "Gobstopper"                  "Haribo Gold Bears"          
## [11] "Haribo Sour Bears"           "Haribo Twin Snakes"         
## [13] "Jawbusters"                  "Laffy Taffy"                
## [15] "Lemonhead"                   "Lifesavers big ring gummies"
## [17] "Mike & Ike"                  "Nerds"                      
## [19] "Nik L Nip"                   "Now & Later"                
## [21] "Pop Rocks"                   "Red vines"                  
## [23] "Ring pop"                    "Runts"                      
## [25] "Skittles original"           "Skittles wildberry"         
## [27] "Smarties candy"              "Sour Patch Kids"            
## [29] "Sour Patch Tricksters"       "Starburst"                  
## [31] "Strawberry bon bons"         "Super Bubble"               
## [33] "Swedish Fish"                "Tootsie Pop"                
## [35] "Trolli Sour Bites"           "Twizzlers"                  
## [37] "Warheads"                    "Welch's Fruit Snacks"
\end{verbatim}

\begin{quote}
Q3. What is your favorite candy in the dataset and what is it's
winpercent value? In other words, how often does my favorite candy win?
\end{quote}

\begin{Shaded}
\begin{Highlighting}[]
\NormalTok{candy[}\StringTok{"Snickers"}\NormalTok{, ]}\SpecialCharTok{$}\NormalTok{winpercent}
\end{Highlighting}
\end{Shaded}

\begin{verbatim}
## [1] 76.67378
\end{verbatim}

\begin{quote}
Q4. What is the winpercent value for ``Kit Kat''?
\end{quote}

\begin{Shaded}
\begin{Highlighting}[]
\NormalTok{candy[}\StringTok{"Kit Kat"}\NormalTok{, ]}\SpecialCharTok{$}\NormalTok{winpercent}
\end{Highlighting}
\end{Shaded}

\begin{verbatim}
## [1] 76.7686
\end{verbatim}

\begin{quote}
Q5. What is the winpercent value for ``Tootsie Roll Snack Bars''?
\end{quote}

\begin{Shaded}
\begin{Highlighting}[]
\NormalTok{candy[}\StringTok{"Tootsie Roll Snack Bars"}\NormalTok{, ]}\SpecialCharTok{$}\NormalTok{winpercent}
\end{Highlighting}
\end{Shaded}

\begin{verbatim}
## [1] 49.6535
\end{verbatim}

\begin{Shaded}
\begin{Highlighting}[]
\CommentTok{\#install.packages(\textquotesingle{}skimr\textquotesingle{})}
\FunctionTok{library}\NormalTok{(}\StringTok{"skimr"}\NormalTok{)}
\end{Highlighting}
\end{Shaded}

\begin{verbatim}
## Warning: package 'skimr' was built under R version 4.2.3
\end{verbatim}

\begin{Shaded}
\begin{Highlighting}[]
\FunctionTok{skim}\NormalTok{(candy)}
\end{Highlighting}
\end{Shaded}

\begin{longtable}[]{@{}ll@{}}
\caption{Data summary}\tabularnewline
\toprule\noalign{}
\endfirsthead
\endhead
\bottomrule\noalign{}
\endlastfoot
Name & candy \\
Number of rows & 85 \\
Number of columns & 12 \\
\_\_\_\_\_\_\_\_\_\_\_\_\_\_\_\_\_\_\_\_\_\_\_ & \\
Column type frequency: & \\
numeric & 12 \\
\_\_\_\_\_\_\_\_\_\_\_\_\_\_\_\_\_\_\_\_\_\_\_\_ & \\
Group variables & None \\
\end{longtable}

\textbf{Variable type: numeric}

\begin{longtable}[]{@{}
  >{\raggedright\arraybackslash}p{(\columnwidth - 20\tabcolsep) * \real{0.1910}}
  >{\raggedleft\arraybackslash}p{(\columnwidth - 20\tabcolsep) * \real{0.1124}}
  >{\raggedleft\arraybackslash}p{(\columnwidth - 20\tabcolsep) * \real{0.1573}}
  >{\raggedleft\arraybackslash}p{(\columnwidth - 20\tabcolsep) * \real{0.0674}}
  >{\raggedleft\arraybackslash}p{(\columnwidth - 20\tabcolsep) * \real{0.0674}}
  >{\raggedleft\arraybackslash}p{(\columnwidth - 20\tabcolsep) * \real{0.0674}}
  >{\raggedleft\arraybackslash}p{(\columnwidth - 20\tabcolsep) * \real{0.0674}}
  >{\raggedleft\arraybackslash}p{(\columnwidth - 20\tabcolsep) * \real{0.0674}}
  >{\raggedleft\arraybackslash}p{(\columnwidth - 20\tabcolsep) * \real{0.0674}}
  >{\raggedleft\arraybackslash}p{(\columnwidth - 20\tabcolsep) * \real{0.0674}}
  >{\raggedright\arraybackslash}p{(\columnwidth - 20\tabcolsep) * \real{0.0674}}@{}}
\toprule\noalign{}
\begin{minipage}[b]{\linewidth}\raggedright
skim\_variable
\end{minipage} & \begin{minipage}[b]{\linewidth}\raggedleft
n\_missing
\end{minipage} & \begin{minipage}[b]{\linewidth}\raggedleft
complete\_rate
\end{minipage} & \begin{minipage}[b]{\linewidth}\raggedleft
mean
\end{minipage} & \begin{minipage}[b]{\linewidth}\raggedleft
sd
\end{minipage} & \begin{minipage}[b]{\linewidth}\raggedleft
p0
\end{minipage} & \begin{minipage}[b]{\linewidth}\raggedleft
p25
\end{minipage} & \begin{minipage}[b]{\linewidth}\raggedleft
p50
\end{minipage} & \begin{minipage}[b]{\linewidth}\raggedleft
p75
\end{minipage} & \begin{minipage}[b]{\linewidth}\raggedleft
p100
\end{minipage} & \begin{minipage}[b]{\linewidth}\raggedright
hist
\end{minipage} \\
\midrule\noalign{}
\endhead
\bottomrule\noalign{}
\endlastfoot
chocolate & 0 & 1 & 0.44 & 0.50 & 0.00 & 0.00 & 0.00 & 1.00 & 1.00 &
▇▁▁▁▆ \\
fruity & 0 & 1 & 0.45 & 0.50 & 0.00 & 0.00 & 0.00 & 1.00 & 1.00 &
▇▁▁▁▆ \\
caramel & 0 & 1 & 0.16 & 0.37 & 0.00 & 0.00 & 0.00 & 0.00 & 1.00 &
▇▁▁▁▂ \\
peanutyalmondy & 0 & 1 & 0.16 & 0.37 & 0.00 & 0.00 & 0.00 & 0.00 & 1.00
& ▇▁▁▁▂ \\
nougat & 0 & 1 & 0.08 & 0.28 & 0.00 & 0.00 & 0.00 & 0.00 & 1.00 &
▇▁▁▁▁ \\
crispedricewafer & 0 & 1 & 0.08 & 0.28 & 0.00 & 0.00 & 0.00 & 0.00 &
1.00 & ▇▁▁▁▁ \\
hard & 0 & 1 & 0.18 & 0.38 & 0.00 & 0.00 & 0.00 & 0.00 & 1.00 & ▇▁▁▁▂ \\
bar & 0 & 1 & 0.25 & 0.43 & 0.00 & 0.00 & 0.00 & 0.00 & 1.00 & ▇▁▁▁▂ \\
pluribus & 0 & 1 & 0.52 & 0.50 & 0.00 & 0.00 & 1.00 & 1.00 & 1.00 &
▇▁▁▁▇ \\
sugarpercent & 0 & 1 & 0.48 & 0.28 & 0.01 & 0.22 & 0.47 & 0.73 & 0.99 &
▇▇▇▇▆ \\
pricepercent & 0 & 1 & 0.47 & 0.29 & 0.01 & 0.26 & 0.47 & 0.65 & 0.98 &
▇▇▇▇▆ \\
winpercent & 0 & 1 & 50.32 & 14.71 & 22.45 & 39.14 & 47.83 & 59.86 &
84.18 & ▃▇▆▅▂ \\
\end{longtable}

\begin{quote}
Q6. Is there any variable/column that looks to be on a different scale
to the majority of the other columns in the dataset?
\end{quote}

Yes, the \texttt{winpercent} column is on a 0:100 scale and all others
appear to be on a 0:1 scale.

\begin{quote}
Q7. What do you think a zero and one represent for the candy\$chocolate
column?
\end{quote}

A zero here means the candy is not classified as containing chocolate.

\begin{quote}
Q8. Plot a histogram of winpercent values
\end{quote}

\begin{Shaded}
\begin{Highlighting}[]
\FunctionTok{hist}\NormalTok{(candy}\SpecialCharTok{$}\NormalTok{winpercent)}
\end{Highlighting}
\end{Shaded}

\includegraphics{halloween_candy_files/figure-latex/unnamed-chunk-9-1.pdf}

\begin{Shaded}
\begin{Highlighting}[]
\FunctionTok{library}\NormalTok{(ggplot2)}
\end{Highlighting}
\end{Shaded}

\begin{verbatim}
## Warning: package 'ggplot2' was built under R version 4.3.0
\end{verbatim}

\begin{Shaded}
\begin{Highlighting}[]
\FunctionTok{ggplot}\NormalTok{(candy) }\SpecialCharTok{+}
  \FunctionTok{aes}\NormalTok{(winpercent) }\SpecialCharTok{+}
  \FunctionTok{geom\_histogram}\NormalTok{(}\AttributeTok{bins=}\DecValTok{10}\NormalTok{)}
\end{Highlighting}
\end{Shaded}

\includegraphics{halloween_candy_files/figure-latex/unnamed-chunk-10-1.pdf}
\textgreater{} Q9. Is the distribution of winpercent values symmetrical?

No.

\begin{quote}
Q10. Is the center of the distribution above or below 50\%?
\end{quote}

Below 50\% with a mean of:

\begin{Shaded}
\begin{Highlighting}[]
\FunctionTok{mean}\NormalTok{(candy}\SpecialCharTok{$}\NormalTok{winpercent)}
\end{Highlighting}
\end{Shaded}

\begin{verbatim}
## [1] 50.31676
\end{verbatim}

\begin{quote}
Q11. On average is chocolate candy higher or lower ranked than fruit
candy?
\end{quote}

To answer this, I need to subset the candy dataset to select only
chocolate candy and then calculate the mean of these. Then, I need to do
the same for fruity candy.

\begin{Shaded}
\begin{Highlighting}[]
\CommentTok{\# Filter/subset data to just get chocolate rows}
\NormalTok{chocolate.candy }\OtherTok{\textless{}{-}}\NormalTok{ candy[}\FunctionTok{as.logical}\NormalTok{(candy}\SpecialCharTok{$}\NormalTok{chocolate), ]}

\CommentTok{\# Get winpercent values}
\NormalTok{chocolate.winpercent }\OtherTok{\textless{}{-}}\NormalTok{ chocolate.candy}\SpecialCharTok{$}\NormalTok{winpercent}

\CommentTok{\# Calculate the mean of winpercent values}
\FunctionTok{mean}\NormalTok{(chocolate.winpercent)}
\end{Highlighting}
\end{Shaded}

\begin{verbatim}
## [1] 60.92153
\end{verbatim}

\begin{Shaded}
\begin{Highlighting}[]
\CommentTok{\# Filter/subset data to just get fruity rows}
\NormalTok{fruity.candy }\OtherTok{\textless{}{-}}\NormalTok{ candy[}\FunctionTok{as.logical}\NormalTok{(candy}\SpecialCharTok{$}\NormalTok{fruity), ]}

\CommentTok{\# Get winpercent values}
\NormalTok{fruity.winpercent }\OtherTok{\textless{}{-}}\NormalTok{ fruity.candy}\SpecialCharTok{$}\NormalTok{winpercent}

\CommentTok{\# Calculate the mean of winpercent values}
\FunctionTok{mean}\NormalTok{(fruity.winpercent)}
\end{Highlighting}
\end{Shaded}

\begin{verbatim}
## [1] 44.11974
\end{verbatim}

So, chocolate (60.92\%) is more popular than fruity (44.11\%).

\begin{quote}
Q12. Is this difference statistically significant?
\end{quote}

Yes, the difference is large, so it is probably statistically
significant.

\hypertarget{overall-candy-ranking}{%
\subsection{Overall Candy Ranking}\label{overall-candy-ranking}}

There is a base R function called \texttt{sort()} for sorting vectors of
input.

\begin{Shaded}
\begin{Highlighting}[]
\NormalTok{x }\OtherTok{\textless{}{-}} \FunctionTok{c}\NormalTok{(}\DecValTok{5}\NormalTok{, }\DecValTok{2}\NormalTok{, }\DecValTok{10}\NormalTok{)}

\FunctionTok{sort}\NormalTok{(x, }\AttributeTok{decreasing =} \ConstantTok{TRUE}\NormalTok{)}
\end{Highlighting}
\end{Shaded}

\begin{verbatim}
## [1] 10  5  2
\end{verbatim}

There is a function related to \texttt{sort()} called \texttt{order()},
which gives us the order of indices.

\begin{Shaded}
\begin{Highlighting}[]
\FunctionTok{order}\NormalTok{(x)}
\end{Highlighting}
\end{Shaded}

\begin{verbatim}
## [1] 2 1 3
\end{verbatim}

\begin{Shaded}
\begin{Highlighting}[]
\NormalTok{x[ }\FunctionTok{order}\NormalTok{(x) ]}
\end{Highlighting}
\end{Shaded}

\begin{verbatim}
## [1]  2  5 10
\end{verbatim}

\begin{quote}
Q13. What are the five least liked candy types in this set?
\end{quote}

\begin{Shaded}
\begin{Highlighting}[]
\NormalTok{ord }\OtherTok{\textless{}{-}} \FunctionTok{order}\NormalTok{(candy}\SpecialCharTok{$}\NormalTok{winpercent)}
\FunctionTok{head}\NormalTok{(candy[ord,])}
\end{Highlighting}
\end{Shaded}

\begin{verbatim}
##                    chocolate fruity caramel peanutyalmondy nougat
## Nik L Nip                  0      1       0              0      0
## Boston Baked Beans         0      0       0              1      0
## Chiclets                   0      1       0              0      0
## Super Bubble               0      1       0              0      0
## Jawbusters                 0      1       0              0      0
## Root Beer Barrels          0      0       0              0      0
##                    crispedricewafer hard bar pluribus sugarpercent pricepercent
## Nik L Nip                         0    0   0        1        0.197        0.976
## Boston Baked Beans                0    0   0        1        0.313        0.511
## Chiclets                          0    0   0        1        0.046        0.325
## Super Bubble                      0    0   0        0        0.162        0.116
## Jawbusters                        0    1   0        1        0.093        0.511
## Root Beer Barrels                 0    1   0        1        0.732        0.069
##                    winpercent
## Nik L Nip            22.44534
## Boston Baked Beans   23.41782
## Chiclets             24.52499
## Super Bubble         27.30386
## Jawbusters           28.12744
## Root Beer Barrels    29.70369
\end{verbatim}

\begin{quote}
Q14. What are the top 5 all time favorite candy types out of this set?
\end{quote}

\begin{Shaded}
\begin{Highlighting}[]
\NormalTok{ord }\OtherTok{\textless{}{-}} \FunctionTok{order}\NormalTok{(candy}\SpecialCharTok{$}\NormalTok{winpercent, }\AttributeTok{decreasing =} \ConstantTok{TRUE}\NormalTok{)}
\FunctionTok{head}\NormalTok{(candy[ord,])}
\end{Highlighting}
\end{Shaded}

\begin{verbatim}
##                           chocolate fruity caramel peanutyalmondy nougat
## Reese's Peanut Butter cup         1      0       0              1      0
## Reese's Miniatures                1      0       0              1      0
## Twix                              1      0       1              0      0
## Kit Kat                           1      0       0              0      0
## Snickers                          1      0       1              1      1
## Reese's pieces                    1      0       0              1      0
##                           crispedricewafer hard bar pluribus sugarpercent
## Reese's Peanut Butter cup                0    0   0        0        0.720
## Reese's Miniatures                       0    0   0        0        0.034
## Twix                                     1    0   1        0        0.546
## Kit Kat                                  1    0   1        0        0.313
## Snickers                                 0    0   1        0        0.546
## Reese's pieces                           0    0   0        1        0.406
##                           pricepercent winpercent
## Reese's Peanut Butter cup        0.651   84.18029
## Reese's Miniatures               0.279   81.86626
## Twix                             0.906   81.64291
## Kit Kat                          0.511   76.76860
## Snickers                         0.651   76.67378
## Reese's pieces                   0.651   73.43499
\end{verbatim}

\begin{quote}
Q15. Make a first barplot of candy ranking based on winpercent values.
\end{quote}

\begin{Shaded}
\begin{Highlighting}[]
\FunctionTok{ggplot}\NormalTok{(candy) }\SpecialCharTok{+}
  \FunctionTok{aes}\NormalTok{(winpercent, }\FunctionTok{reorder}\NormalTok{( }\FunctionTok{rownames}\NormalTok{(candy), winpercent)) }\SpecialCharTok{+}
  \FunctionTok{geom\_col}\NormalTok{()}
\end{Highlighting}
\end{Shaded}

\includegraphics{halloween_candy_files/figure-latex/unnamed-chunk-18-1.pdf}

\begin{Shaded}
\begin{Highlighting}[]
\NormalTok{my\_cols}\OtherTok{=}\FunctionTok{rep}\NormalTok{(}\StringTok{"black"}\NormalTok{, }\FunctionTok{nrow}\NormalTok{(candy))}
\NormalTok{my\_cols[}\FunctionTok{as.logical}\NormalTok{(candy}\SpecialCharTok{$}\NormalTok{chocolate)] }\OtherTok{=} \StringTok{"chocolate"}
\NormalTok{my\_cols[}\FunctionTok{as.logical}\NormalTok{(candy}\SpecialCharTok{$}\NormalTok{bar)] }\OtherTok{=} \StringTok{"brown"}
\NormalTok{my\_cols[}\FunctionTok{as.logical}\NormalTok{(candy}\SpecialCharTok{$}\NormalTok{fruity)] }\OtherTok{=} \StringTok{"pink"}
\end{Highlighting}
\end{Shaded}

\begin{Shaded}
\begin{Highlighting}[]
\FunctionTok{ggplot}\NormalTok{(candy) }\SpecialCharTok{+} 
  \FunctionTok{aes}\NormalTok{(winpercent, }\FunctionTok{reorder}\NormalTok{(}\FunctionTok{rownames}\NormalTok{(candy),winpercent)) }\SpecialCharTok{+}
  \FunctionTok{geom\_col}\NormalTok{(}\AttributeTok{fill=}\NormalTok{my\_cols) }
\end{Highlighting}
\end{Shaded}

\includegraphics{halloween_candy_files/figure-latex/unnamed-chunk-20-1.pdf}
\textgreater{} Q17. What is the worst ranked chocolate candy?

Sixlets

\begin{quote}
Q18. What is the best ranked fruit candy?
\end{quote}

Starburst

\hypertarget{taking-a-look-at-pricepercent}{%
\subsection{Taking a look at
pricepercent}\label{taking-a-look-at-pricepercent}}

\begin{Shaded}
\begin{Highlighting}[]
\NormalTok{my\_cols[}\FunctionTok{as.logical}\NormalTok{(candy}\SpecialCharTok{$}\NormalTok{fruity)]}
\end{Highlighting}
\end{Shaded}

\begin{verbatim}
##  [1] "pink" "pink" "pink" "pink" "pink" "pink" "pink" "pink" "pink" "pink"
## [11] "pink" "pink" "pink" "pink" "pink" "pink" "pink" "pink" "pink" "pink"
## [21] "pink" "pink" "pink" "pink" "pink" "pink" "pink" "pink" "pink" "pink"
## [31] "pink" "pink" "pink" "pink" "pink" "pink" "pink" "pink"
\end{verbatim}

\begin{Shaded}
\begin{Highlighting}[]
\FunctionTok{ggplot}\NormalTok{(candy) }\SpecialCharTok{+}
  \FunctionTok{aes}\NormalTok{(winpercent, pricepercent, }\AttributeTok{label=}\FunctionTok{rownames}\NormalTok{(candy)) }\SpecialCharTok{+}
  \FunctionTok{geom\_point}\NormalTok{(}\AttributeTok{col=}\NormalTok{my\_cols)}
\end{Highlighting}
\end{Shaded}

\includegraphics{halloween_candy_files/figure-latex/unnamed-chunk-22-1.pdf}
\#\#\# Add some labels

\begin{Shaded}
\begin{Highlighting}[]
\FunctionTok{ggplot}\NormalTok{(candy) }\SpecialCharTok{+}
  \FunctionTok{aes}\NormalTok{(winpercent, pricepercent, }\AttributeTok{label =} \FunctionTok{rownames}\NormalTok{(candy)) }\SpecialCharTok{+}
  \FunctionTok{geom\_point}\NormalTok{(}\AttributeTok{col =}\NormalTok{ my\_cols) }\SpecialCharTok{+}
  \FunctionTok{geom\_text}\NormalTok{()}
\end{Highlighting}
\end{Shaded}

\includegraphics{halloween_candy_files/figure-latex/unnamed-chunk-23-1.pdf}

\begin{Shaded}
\begin{Highlighting}[]
\CommentTok{\#install.packages("ggrepel")}
\FunctionTok{library}\NormalTok{(ggrepel)}
\end{Highlighting}
\end{Shaded}

\begin{verbatim}
## Warning: package 'ggrepel' was built under R version 4.2.3
\end{verbatim}

\begin{Shaded}
\begin{Highlighting}[]
\CommentTok{\# This will give us a cleaner graph}

\CommentTok{\# How about a plot of pricepercent vs winpercent}
\FunctionTok{ggplot}\NormalTok{(candy) }\SpecialCharTok{+}
  \FunctionTok{aes}\NormalTok{(winpercent, pricepercent, }\AttributeTok{label=}\FunctionTok{rownames}\NormalTok{(candy)) }\SpecialCharTok{+}
  \FunctionTok{geom\_point}\NormalTok{(}\AttributeTok{col=}\NormalTok{my\_cols) }\SpecialCharTok{+} 
  \FunctionTok{geom\_text\_repel}\NormalTok{(}\AttributeTok{col=}\NormalTok{my\_cols, }\AttributeTok{size=}\FloatTok{3.3}\NormalTok{, }\AttributeTok{max.overlaps =} \DecValTok{5}\NormalTok{)}
\end{Highlighting}
\end{Shaded}

\begin{verbatim}
## Warning: ggrepel: 54 unlabeled data points (too many overlaps). Consider
## increasing max.overlaps
\end{verbatim}

\includegraphics{halloween_candy_files/figure-latex/unnamed-chunk-24-1.pdf}

\hypertarget{exploring-the-correlation-structure}{%
\subsection{Exploring the correlation
structure}\label{exploring-the-correlation-structure}}

Pearson correlation goes between -1 and +1 with zero indicating no
correlation and values close to one being very highly correlated.

\begin{Shaded}
\begin{Highlighting}[]
\FunctionTok{library}\NormalTok{(corrplot)}
\end{Highlighting}
\end{Shaded}

\begin{verbatim}
## Warning: package 'corrplot' was built under R version 4.2.3
\end{verbatim}

\begin{verbatim}
## corrplot 0.92 loaded
\end{verbatim}

\begin{Shaded}
\begin{Highlighting}[]
\NormalTok{cij }\OtherTok{\textless{}{-}} \FunctionTok{cor}\NormalTok{(candy)}
\FunctionTok{corrplot}\NormalTok{(cij)}
\end{Highlighting}
\end{Shaded}

\includegraphics{halloween_candy_files/figure-latex/unnamed-chunk-26-1.pdf}
\textgreater{} Q22. Examining this plot, what two variables are
anti-correlated?

Chocolate and fruit and anti-correlated.

\begin{quote}
Q23. What two variables are most positively correlated?
\end{quote}

Chocolate and winpercent or chocolate and bar are the most positively
correlated.

\hypertarget{principal-component-analysis}{%
\section{6. Principal Component
Analysis}\label{principal-component-analysis}}

The base R function for PCA is called \texttt{prcomp()} and we can set
``scale=TRUE/FALSE''.

\begin{Shaded}
\begin{Highlighting}[]
\NormalTok{pca }\OtherTok{\textless{}{-}} \FunctionTok{prcomp}\NormalTok{(candy, }\AttributeTok{scale =} \ConstantTok{TRUE}\NormalTok{)}

\FunctionTok{summary}\NormalTok{(pca)}
\end{Highlighting}
\end{Shaded}

\begin{verbatim}
## Importance of components:
##                           PC1    PC2    PC3     PC4    PC5     PC6     PC7
## Standard deviation     2.0788 1.1378 1.1092 1.07533 0.9518 0.81923 0.81530
## Proportion of Variance 0.3601 0.1079 0.1025 0.09636 0.0755 0.05593 0.05539
## Cumulative Proportion  0.3601 0.4680 0.5705 0.66688 0.7424 0.79830 0.85369
##                            PC8     PC9    PC10    PC11    PC12
## Standard deviation     0.74530 0.67824 0.62349 0.43974 0.39760
## Proportion of Variance 0.04629 0.03833 0.03239 0.01611 0.01317
## Cumulative Proportion  0.89998 0.93832 0.97071 0.98683 1.00000
\end{verbatim}

The main result of PCA - i.e.~the new PC plot (projection of candy on
our new PC axis) is contained in \texttt{pca\$x}.

\begin{Shaded}
\begin{Highlighting}[]
\NormalTok{pc }\OtherTok{\textless{}{-}} \FunctionTok{as.data.frame}\NormalTok{(pca}\SpecialCharTok{$}\NormalTok{x)}

\FunctionTok{ggplot}\NormalTok{(pc) }\SpecialCharTok{+} 
  \FunctionTok{aes}\NormalTok{(PC1, PC2, }\AttributeTok{label =} \FunctionTok{rownames}\NormalTok{(pc)) }\SpecialCharTok{+}
  \FunctionTok{geom\_point}\NormalTok{(}\AttributeTok{col =}\NormalTok{ my\_cols)}
\end{Highlighting}
\end{Shaded}

\includegraphics{halloween_candy_files/figure-latex/unnamed-chunk-28-1.pdf}

\begin{quote}
Q24. What original variables are picked up strongly by PC1 in the
positive direction? Do these make sense to you?
\end{quote}

Fruity, hard, and pluribus

\begin{Shaded}
\begin{Highlighting}[]
\FunctionTok{par}\NormalTok{(}\AttributeTok{mar=}\FunctionTok{c}\NormalTok{(}\DecValTok{8}\NormalTok{, }\DecValTok{4}\NormalTok{, }\DecValTok{2}\NormalTok{, }\DecValTok{2}\NormalTok{))}
\FunctionTok{barplot}\NormalTok{(pca}\SpecialCharTok{$}\NormalTok{rotation[,}\DecValTok{1}\NormalTok{], }\AttributeTok{las=}\DecValTok{2}\NormalTok{, }\AttributeTok{ylab=}\StringTok{"PC1 Contribution"}\NormalTok{)}
\end{Highlighting}
\end{Shaded}

\includegraphics{halloween_candy_files/figure-latex/unnamed-chunk-29-1.pdf}

\end{document}
